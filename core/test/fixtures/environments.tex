** nested horizontal environments
.
This \begin{empty}is a \begin{empty}line\end{empty} of\end{empty} text.
.


** itemize environment
.
a paragraph of text

\begin{itemize}
    \item first item

        with a new paragraph
    \item second item
\end{itemize}
%

more text

\begin{itemize}
    \item first item
    \item second item

\end{itemize}
this belongs to paragraph with itemize
.


** empty and consecutive itemize environments
.
\begin{itemize}

\end{itemize}
\begin{itemize}
    \item text that doesn't continue
\end{itemize}
\begin{itemize}
\end{itemize}
.


s** nested itemize environments
.
text before
\begin{itemize}
    \item first item
        \begin{itemize}
            \item first nested item
        \end{itemize}
\end{itemize}
text after
.


s** nested itemize environments (II)
.
text before
\begin{itemize}
    \item first item
        \begin{itemize}
            \item first nested item
        \end{itemize}
    \item second item
\end{itemize}
text after
.


s** single itemize environment, custom labels
.
\begin{itemize}
    \item[\textendash] first item
    \item[] second item
    \item third item
\end{itemize}
.


s** enumerate environment
.
a paragraph of text

\begin{enumerate}
    \item[\itshape label] first item

        with a new paragraph
    \item second item
\end{enumerate}
this belongs to paragraph with enumerate
.


s** description environment
.
a paragraph of text

\begin{description}
    \item[term 1] first item

        with a new paragraph
    \item[term 2] second item
\end{description}
this belongs to paragraph with description
.


s** quote, quotation, verse
.
\noindent quote:
\begin{quote}
    Here are some quote\\
    lines of text

    as well as a paragraph with special chars [].
\end{quote}
quotation:
\begin{quotation}
    Here are some quotation\\
    lines of text

    as well as a paragraph.
\end{quotation}
verse:
\begin{verse}
    Here are some verse lines of text lines of text lines of text lines of text lines of text \\
    lines of text lines of text lines of text lines of text lines of text lines of text

    as well as a paragraph with trailing newline.\\
\end{verse}
And now we continue without indentation.
.


s** font environments
.
normal text \begin{small}
    small text
    \begin{bfseries}
        bold text
    \end{bfseries}
\end{small}
  three spaces!
.


s** alignment
.
text
\begin{center}
    Here are some centered\\
    lines of text

    as well as a paragraph.
\end{center}
Afterwards, we continue as usual.
\begin{flushleft}
    The same is true\\
    for a flushleft environment.
\end{flushleft}
Afterwards, we continue as usual.
\begin{flushright}
    The same is true\\
    for a flushright environment.
\end{flushright}

Yes, this is the last line.
.


s** alignment of lists
.
\begin{center}
    Center environment should not center the list.
    \begin{itemize}
        \item first item
        \item second item
    \end{itemize}
\end{center}
But centering command should:
\begin{itemize}
    \centering
    \item[\textendash] first item
    \item[] second item
    \item third item
\end{itemize}
Afterwards, we continue as usual.
.


!** abstract and fonts
.
\begin{abstract}
\itshape
The title must not be italic.
\end{abstract}
.


** verbatim environment
.
\begin{verbatim}
Verbatim line with \TeX{} % inside
\end{verbatim}
.


** starred verbatim environment
.
\begin{verbatim*}
Spaces  are   shown   as   visible   spaces
\end{verbatim*}
.


** empty verbatim environment
.
\begin{verbatim}
\end{verbatim}
.


** verbatim with special characters
.
\begin{verbatim}
Special chars: # $ % ^ & _ { } ~ \
More text with "quotes" and 'apostrophes'
\end{verbatim}
.


** comment environment
.
text
\begin{comment}
    This is a comment.
    \end{comment
    still more comment.
\end{comment}
more text, but now with%
\begin{comment}
    This is a comment.
\end{comment}
out space.
.
