** basic theorem with plain style
.
\documentclass{article}
\usepackage{amsthm}

\newtheorem{theorem}{Theorem}

\begin{document}
\begin{theorem}
This is a theorem statement.
\end{theorem}
\end{document}
.


** theorem with proof
.
\documentclass{article}
\usepackage{amsthm}

\newtheorem{theorem}{Theorem}

\begin{document}
\begin{theorem}[Pythagorean Theorem]
For a right triangle with legs $a$ and $b$, and hypotenuse $c$: $a^2 + b^2 = c^2$.
\end{theorem}

\begin{proof}
This follows from the law of cosines.
\end{proof}
\end{document}
.


** multiple theorem styles
.
\documentclass{article}
\usepackage{amsthm}

\theoremstyle{plain}
\newtheorem{theorem}{Theorem}
\newtheorem{lemma}[theorem]{Lemma}

\theoremstyle{definition}
\newtheorem{definition}{Definition}
\newtheorem{example}{Example}

\theoremstyle{remark}
\newtheorem{remark}{Remark}

\begin{document}
\begin{theorem}
This is a theorem in plain style.
\end{theorem}

\begin{lemma}
This is a lemma sharing the theorem counter.
\end{lemma}

\begin{definition}
This is a definition in definition style.
\end{definition}

\begin{example}
This is an example in definition style.
\end{example}

\begin{remark}
This is a remark in remark style.
\end{remark}
\end{document}
.


** unnumbered theorem
.
\documentclass{article}
\usepackage{amsthm}

\newtheorem*{observation}{Observation}

\begin{document}
\begin{observation}
This theorem is unnumbered.
\end{observation}
\end{document}
.


** sectioned numbering
.
\documentclass{article}
\usepackage{amsthm}

\newtheorem{theorem}{Theorem}[section]

\begin{document}
\section{Introduction}
\begin{theorem}
First theorem in section 1.
\end{theorem}

\begin{theorem}
Second theorem in section 1.
\end{theorem}

\section{Main Results}
\begin{theorem}
First theorem in section 2.
\end{theorem}
\end{document}
.


** proof with custom label
.
\documentclass{article}
\usepackage{amsthm}

\begin{document}
\begin{proof}[Proof Sketch]
This is just a sketch of the proof.
\end{proof}

\begin{proof}[Solution]
This is the solution to the problem.
\end{proof}
\end{document}
.


** qed command
.
\documentclass{article}
\usepackage{amsthm}

\begin{document}
This completes the argument. \qed

Manual QED placement in the middle of text.
\end{document}
.